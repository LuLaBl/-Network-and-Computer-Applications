\documentclass[review]{elsarticle}

\usepackage{lineno,hyperref}
\modulolinenumbers[5]

\journal{Journal of Networks and Computer Applications}

%%%%%%%%%%%%%%%%%%%%%%%
%% Elsevier bibliography styles
%%%%%%%%%%%%%%%%%%%%%%%
%% To change the style, put a % in front of the second line of the current style and
%% remove the % from the second line of the style you would like to use.
%%%%%%%%%%%%%%%%%%%%%%%

%% Numbered
%\bibliographystyle{model1-num-names}

%% Numbered without titles
%\bibliographystyle{model1a-num-names}

%% Harvard
%\bibliographystyle{model2-names.bst}\biboptions{authoryear}

%% Vancouver numbered
%\usepackage{numcompress}\bibliographystyle{model3-num-names}

%% Vancouver name/year
%\usepackage{numcompress}\bibliographystyle{model4-names}\biboptions{authoryear}

%% APA style
%\bibliographystyle{model5-names}\biboptions{authoryear}

%% AMA style
%\usepackage{numcompress}\bibliographystyle{model6-num-names}

%% `Elsevier LaTeX' style
\bibliographystyle{elsarticle-num}
%%%%%%%%%%%%%%%%%%%%%%%

\begin{document}

\begin{frontmatter}

\title{A wearable fall-detection system based on Body Area Networks for smart cities}

%% Group authors per affiliation:
\author{Luigi La Blunda\fnref{label1}}
\ead{l.lablunda@fb2.fra-uas.de (corresponding author)}
\author{Lorena Guti\'errez-Madro\~nal\fnref{label2}}
%% \author{Name\corref{cor1}\fnref{label2}}
\ead{lorena.gutierrez@uca.es}
\author{Matthias F. Wagner\fnref{label1}}
\ead{mfwagner@fb2.fra-uas.de}
\author{I. Medina-Bulo\fnref{label2}}
\ead{inmaculada.medina@uca.es}
\address[label1]{WSN and IOT Research Group Frankfurt University of Applied Sciences, Nibelungenplatz 1, 60318 Frankfurt am Main, Germany}
\address[label2]{UCASE Software Engineering Research group, University of Cadiz}



%% or include affiliations in footnotes:
%\author[mymainaddress,mysecondaryaddress]{Elsevier Inc}
%\ead[url]{www.elsevier.com}
%
%\author[mysecondaryaddress]{Global Customer Service\corref{mycorrespondingauthor}}
%\cortext[mycorrespondingauthor]{Corresponding author}
%\ead{support@elsevier.com}
%
%\address[mymainaddress]{1600 John F Kennedy Boulevard, Philadelphia}
%\address[mysecondaryaddress]{360 Park Avenue South, New York}

\begin{abstract}
Falls can have serious consequences for people, which can lead, for example, to restrictions in mobility or in the worst case to traumatic based cases of death. To provide rapid assistance, a portable fall detection system has been developed which is capable of detecting fall situations and, if necessary, alerting the emergency services without any user interaction. The prototype was designed to facilitate a reliable fall-detection and to classify several fall-types. This solution represents a life-saving service for every inhabitant which would significantly enrich the development of smart cities and smart factories where fall-events are part of daily-life. This paper will also introduce the fall analysis, which includes the generation of test events. To guarantee functional safety, the hazard analysis method STAMP (System-Theroetic Accident Model and Processes) will be applied. 
\end{abstract}

\begin{keyword}
e-Health \sep fall-detection\sep Body Area Network \sep safety \sep STAMP \sep Smart City \ 
\end{keyword}

\end{frontmatter}

\linenumbers

\section{Introduction}
Fall-detection is gaining in importance not only in aging societies, but also in working society and in daily activities. According to the World Health Organization (WHO), 646,000 fatal falls are estimated to be the second leading cause of accidental or unintentional death worldwide each year. People over 65 suffer the most fatal falls. Another high risk group is children. Taking into consideration their evolving developmental stages, the increasing curiosity to explore the environments or the inadequate adult supervision lead to fall-events \cite{WHO2018}.
In everyday life we are also confronted with the risk of falling, during the night shift in a factory or practicing dangerous sport disciplines can lead to fatal falls. Annually 37.3 million fall-events are severe enough to require medical treatment \cite{WHO2018}.



\section{Related Work}

The author names and affiliations could be formatted in two ways:
\begin{enumerate}[(1)]
\item Group the authors per affiliation.
\item Use footnotes to indicate the affiliations.
\end{enumerate}
See the front matter of this document for examples. You are recommended to conform your choice to the journal you are submitting to.


\section{Background}


\section{Fall-detection system prototype}
	

\subsection{Architecture}


\subsection{Sensor fusion}
ECG

\subsection{Generation of test-events}
Lorena's part

\subsection{Detected problems}

\section{Example application of STAMP as hazard analysis method}

\subsection{Introducing STAMP}

\subsection{STAMP - Hazard analysis}

\section{Conclusion \& Future work}


\section{Bibliography styles}

There are various bibliography styles available. You can select the style of your choice in the preamble of this document. These styles are Elsevier styles based on standard styles like Harvard and Vancouver. Please use Bib\TeX\ to generate your bibliography and include DOIs whenever available.

Here are two sample references: \cite{Feynman1963118,Dirac1953888}.

\section*{References}

\bibliography{mybibfile}

\end{document}